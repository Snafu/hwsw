
SYSTEM C

				entwurfssprachen am bsp system c

\section{Allgemeines}
	warum systemc:

	systemdesign \& verifikation HW / SW
	modellierung von systemblöcken
	überblick über interaktion der funktionsblöcke
	entwicklung von embedd. apps bevor prototyp vorhanden ist
	design/verifikation unabghängig von detailierter HW/SW implementierung

	höherer abstraktionslevel durch system c
		--> einfachere tradeoffanalyse(performance/funktionalität)
		--> einfacheres design/redesign(produktivitätssteigerung)

\section{Geschichte}
\begin{itemize}
 \item 1997 scenic(vorgänger systemC
 \item 1999 open systemC inititive(ARM, Synopsis,...)
 \item 2005 IEEE standard IEEE 1666-2005 systemc2.1
\end{itemize}

	
\section{Sprachkonstrukte}
	systemc = c++ class library
	
	module
		c++ klasse – sc\_module
		grundlegende einheit des design
		hierarchisches design(modul innerhalb von modulen)

	processes(cooperative threading)
funktion
sc\_thread(einfache ausführung)
sc\_cthread(clocked thread, synthetisierbar – veraltet)
sc\_method(wird dauernd ausgeführt, kein wait-status sondern nexttrigger)

	ports/interfaces/channels/exports

	sensitivity, events(notification)
		bei welcher signaländerung wird ein process gestartet

	hardwareorientierte typen(mehrwertige logik, bitvektoren)

	hw-eigenschaften werden auf c++ konzepte abgebildet
	
	processes -> funktionen
	interfaces -> virtuelle funktion
	interrupts -> exception handling
	module -> klasse
	architecture -> 

\section{Beispiele}

	hello world – beispiel
	

\section{Vor- \& Nachteile}
\subsection{Vorteile}
\begin{itemize}
 \item unterstützung von design in versch. Abstraktionslevel
 \item design von HW/SW in gleicher umgebung
 \item leichteres debugging(SW statt HW)
 \item verifikation ist leichter(high level modell referenz)
 \item systemC ist standardisiert
\end{itemize}

\subsection{Nachteile}
\begin{itemize}
 \item grosser sprachumfang(viel ballast)
 \item relativ neues konzept:
 \item tools nicht ausgereift
 \item wenig fachkräfte
 \item wenig vorhandene projekte zur orientierung
\end{itemize}

	tools
synopsis(zb. Platform architect, VCS)
mentor graphics(modelsim)
	
	hw – hersteller
transaction level versionene von ARM11, AMBA

	haupteinsatzgebiet von systemC ist an unis: forschung im vordergrund, nicht entwicklung
	
sw-sprachen

	besonderheiten
		hw-erweiterungen unter verwendung schon vorhandener sprachkonzepte
	vorteile
		schnelle entwicklung

	automatisiertes partitioning

	systemc versucht zu erkennen was syncron, was asyncron
		asyncron -> hardwareorientierte
		syncron -> software 
