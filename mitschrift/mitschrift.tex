
%SYSTEM C

%				entwurfssprachen am bsp system c

\section{Allgemeines}
	Warum SystemC: \\
  \begin{itemize}
  	\item Systemdesign \& Verifikation HW / SW
  	\item Modellierung von Systembl�cken, die getrennt entwickel- und testbar sind
  	\item �berblick �ber Interaktion der Funktionsbl�cke
  	\item Entwicklung von embedd. Apps bevor Prototyp vorhanden ist
  	\item Design/Verifikation unabgh�ngig von detailierter HW / SW Implementierung

  	\item H�herer Abstraktionslevel durch SystemC \\
			--> einfachere Tradeoffanalyse (Performance / Funktionalit�t) \\
			--> einfacheres Design / Redesign (Produktivit�tssteigerung)
  \end{itemize}
  
  Der Entwicklungsprozess durchl�uft eine Designschleife, wobei nachdem ein Design entwickelt wurde nach Fehlern gesucht wird.
  Diese Fehler weden in einem neuen Design behoben, worauf wieder nach Fehlern gesucht wird. Dieser Prozess wiederholt sich, bis keine Fehler mehr erkennbar sind.

\section{Geschichte}
	\begin{itemize}
	 \item 1997 Scenic (Vorg�nger SystemC)
	 \item 1999 Open SystemC Inititive (ARM, Synopsis, ...)
	 \item 2005 IEEE Standard IEEE 1666-2005 SystemC 2.1
	\end{itemize}

	
\section{Sprachkonstrukte}
	SystemC ist eine C++ class library
	
	\subsection*{Module}
		\begin{itemize}
		  \item C++ Klasse sc\_module
			\item Grundlegende Einheit des Designs
			\item Hierarchisches Design (Modul innerhalb von Modulen)
		\end{itemize}

	\subsection*{Processes (cooperative threading)}
		\begin{itemize}
			\item Enthalten die Funktion
			\item sc\_thread f�r einmalige Ausf�hrung
			\item sc\_cthread (clocked thread, synthetisierbar - veraltet)
			\item sc\_method wird dauernd ausgef�hrt, kein wait-Status sondern next-Trigger (Sensitivity List)
			\item Simulation Kernel sorgt f�r Threading / Scheduling beim Testen
		\end{itemize}

	Die Schnittstellen der Module (Ports) verwenden Interfaces um mit Channels zu kommunizieren.
	Channels sind z.B. Buffer, FIFOs, Mutex bzw Semaphoren.
	
	
	Sensitivity List bzw Events (Notifications) regeln,	bei welcher Signal�nderung ein Process ausgef�hrt wird.
	Hardwareorientierte Typen (mehrwertige Logik, Bitvektoren) werden �ber C++ Templates realisiert.\\
	
	
	HW-Eigenschaften werden folgendermassen auf C++ Konzepte abgebildet
	\begin{itemize} 
	  \item Processes -> Funktionen
	  \item Interfaces -> Virtuelle Funktion
	  \item Interrupts -> Exception handling
	  \item Module -> Klasse
	  \item Architecture -> Gesamtdesign
	  \end{itemize}

\section{Beispiele}

	Anhand eines einfachen Hello World - Beispiels wurde die Funktionsweise eines sc\_thread mit Simulation demonstriert.

\section{Vor- \& Nachteile}
\subsection{Vorteile}
\begin{itemize}
 \item Unterst�tzung von Design in verschiedenen Abstraktionsleveln
 \item Design von HW/SW in einheitlicher Umgebung
 \item Abstraktion vereinfacht das Debugging (SW statt HW)
 \item Verifikation wird vereinfacht (high level modell referenz)
 \item SystemC ist standardisiert
\end{itemize}

\subsection{Nachteile}
\begin{itemize}
 \item Grosser Sprachumfang (viel Ballast)
 \item Relativ neues Konzept
 \item Tools noch nicht ausgereift
 \item Wenig Fachkr�fte
 \item Wenig vorhandene Projekte zur Orientierung
\end{itemize}



\subsection{vorhandene Tools}
\begin{itemize}
 \item Platform architect, VCS von Synopsis
 \item Modelsim von Mentor Graphics
 \item Simulation und Verifikation m�glich, Synthese jedoch derzeit nicht
\end{itemize}

\subsection{HW - Hersteller / Realisierte Projekt}
\begin{itemize}
 \item ARM11
 \item AMBA
\end{itemize}

\section{Verwendung}
\begin{itemize}
 \item Haupteinsatzgebiet von SystemC ist an Universit�ten
 \item Forschung im Vordergrund
 \item Entwicklung von kommerziellen Produkten steht nicht im Vordergrund
\end{itemize}

\section{Besonderheiten}
\begin{itemize}
 \item HW-Erweiterungen unter Verwendung schon vorhandener Sprachkonzepte
 \item Automatische Partionierung
 \item Abbildung synchroner Komponenten in Software
 \item Abbildung asynchroner Komponenten in Hardware
\end{itemize}

